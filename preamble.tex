\documentclass[a4paper, 12pt]{article}

\usepackage[english, russian]{babel}
\usepackage[T2A]{fontenc}
\usepackage[utf8]{inputenc}
\usepackage{indentfirst}

\usepackage{amsmath, amsfonts, amssymb, amsthm, mathtools, mathtext}

\theoremstyle{plain}
\newtheorem{theorem}{Теорема}[section]
\newtheorem{lemma}{Лемма}[section]
\newtheorem{mention}{Замечание}[section]
\newtheorem{corollary}{Следствие}[theorem]
\newtheorem{definition}{Определение}[section]
\newtheorem{example}{Пример}[section]
\newtheorem{sentence}{Предложение}[section]
\newtheorem{explanation}{Пояснение}[section]

\usepackage{geometry}
\geometry{top=25mm}
\geometry{bottom=30mm}
\geometry{left=15mm}
\geometry{right=15mm}

\usepackage{titleps}
\newpagestyle{main}{
	\setheadrule{0.4pt}
	\sethead{НИУ ВШЭ}{}{Корчагин Егор}
	\setfootrule{0.4pt}
	\setfoot{Конспект лекций по математическому анализу 2023/2024}{}{\thepage}
}
\pagestyle{main}

\usepackage{soulutf8}

\newcommand{\deriv}[2]{\frac{\partial #1}{\partial #2}}
\newcommand{\R}{\mathbb R}
\newcommand{\Z}{\mathbb Z}
\newcommand{\N}{\mathbb N}
\newcommand{\Q}{\mathbb Q}
\newcommand{\I}{\mathbb I}

\renewcommand{\phi}{\varphi}
\renewcommand{\epsilon}{\varepsilon}
\renewcommand{\kappa}{\varkappa}

\DeclareMathOperator{\Kerr}{Ker}
\DeclareMathOperator{\Imm}{Im}

\DeclareMathOperator*{\argmax}{argmax}

\newcommand{\percent}{\mathbin{\%}}
\newcommand{\symdiff}{\mathbin{\triangle}}
